\chapter*{Memory Map}
\addcontentsline{toc}{chapter}{\protect\numberline{}Memory Map}

The Commander X16 has 512 KB of ROM and 2,088 KB (512 KB\footnote{Developer
editions of the Commander X16 typically come with 2MB of banked RAM rather than
512 MB} + 40 KB) of RAM with up to 3.5MB of RAM or ROM available to
cartridges.\\

Some of the ROM/RAM is always visible at certain address ranges, while the
remaining ROM/RAM is banked into one of two address windows.\\

This is an overview of the Commander X16 memory map:\\

\begin{tabular}{|c|p{0.75\linewidth}|}
	\hline
	{\bfseries Addresses} & {\bfseries Description}\\ \hline
	\$0000-\$9EFF & Fixed RAM (40 KB minus 256 bytes)\\ \hline
	\$9F00-\$9FFF & I/O Area (256 bytes)\\ \hline
	\$A000-\$BFFF & Banked RAM (8 KB window into one of 256 banks for a total of 2 MB)\\ \hline
	\$C000-\$FFFF & Banked System ROM and Cartridge ROM/RAM (16 KB window into one of 256 banks, see below)\\ \hline
\end{tabular}

\section{Banked Memory}

Writing to the following zero-page addresses sets the desired RAM or ROM bank:\\

\begin{tabular}{|c|p{0.75\linewidth}|}
	\hline
	{\bfseries Address} & {\bfseries Description}\\ \hline
	\$0000 & Current RAM bank (0-255)\\ \hline
	\$0001 & Current ROM/Cartridge bank (ROM is 0-31, Cartridge is 32-255)\\ \hline
\end{tabular}

\vspace{16pt}

The currently set banks can also be read back from the respective memory locations. Both settings default to 0 on RESET.\\

\section{ROM Allocations}

Here is the ROM/Cartridge bank allocation:\\

\begin{tabular}{|c|c|p{0.55\linewidth}|}
	\hline
	{\bfseries Bank}   & {\bfseries Name}    & {\bfseries Description}\\ \hline
	0      & KERNAL  & KERNAL operating system and drivers\\ \hline
	1      & KEYBD   & Keyboard layout tables\\ \hline
	2      & CBDOS   & The computer-based CMDR-DOS for FAT32 SD cards\\ \hline
	3      & FAT32   & The FAT32 driver itself\\ \hline
	4      & BASIC   & BASIC interpreter\\ \hline
	5      & MONITOR & Machine Language Monitor\\ \hline
	6      & CHARSET & PETSCII and ISO character sets (uploaded into VRAM)\\ \hline
	7      & CODEX   & CodeX16 Interactive Assembly Environment / Monitor\\ \hline
	8      & GRAPH   & Kernal graphics and font routines\\ \hline
	9      & DEMO    & Demo routines\\ \hline
	10     & AUDIO   & Audio API routines\\ \hline
	11     & UTIL    & System Configuration (Date/Time, Display Preferences)\\ \hline
	12     & BANNEX  & BASIC Annex (code for some added BASIC functions)\\ \hline
	13-14  & X16EDIT & The built-in text editor\\ \hline
	13-31  & –       & (Currently unused)\\ \hline
	32-255 & –       & Cartridge RAM/ROM\\ \hline
\end{tabular}

\vspace{16pt}

\subsection{Cartridge Allocation}

Cartridges can use the remaining 32-255 banks in any combination of ROM, RAM,
Memory-Mapped IO, etc.  This provides up to 3.5MB of additional RAM or ROM.\\

\section{RAM Contents}

This is the allocation of fixed RAM in the KERNAL/BASIC environment.\\

\begin{tabular}{|c|p{0.75\linewidth}|}
	\hline
	{\bfseries Addresses} & {\bfseries Description}\\ \hline
	\$0000-\$00FF & Zero page\\ \hline
	\$0100-\$01FF & CPU stack\\ \hline
	\$0200-\$03FF & KERNAL and BASIC variables, vectors\\ \hline
	\$0400-\$07FF & Available for machine code programs or custom data storage\\ \hline
	\$0800-\$9EFF & BASIC program/variables; available to the user\\ \hline
\end{tabular}

\vspace{16pt}

The \$0400-\$07FF can be seen as the equivalent of \$C000-\$CFFF on a C64. A
typical use would be for helper machine code called by BASIC.\\

\subsection{Zero Page}

\begin{tabular}{|c|p{0.75\linewidth}|}
	\hline
	{\bfseries Addresses} & {\bfseries Description}\\ \hline
	\$0000-\$0001 & Banking registers\\ \hline
	\$0002-\$0021 & 16 bit registers r0-r15 for KERNAL API\\ \hline
	\$0022-\$007F & Available to the user\\ \hline
	\$0080-\$009C & Used by KERNAL and DOS\\ \hline
	\$009D-\$00A8 & Reserved for DOS/BASIC\\ \hline
	\$00A9-\$00D3 & Used by the Math library (and BASIC)\\ \hline
	\$00D4-\$00FF & Used by BASIC\\ \hline
\end{tabular}

\vspace{16pt}

Machine code applications are free to reuse the BASIC area, and if they don't use the Math library, also that area.

\subsection{RAM Banks}

This is the allocation of banked RAM in the KERNAL/BASIC environment.\\

\begin{tabular}{|c|p{0.75\linewidth}|}
	\hline
	{\bfseries Bank} & {\bfseries Description}\\ \hline
	0     & Used for KERNAL/CMDR-DOS variables and buffers\\ \hline
	1-63 & Available to the user\\ \hline
\end{tabular}

\vspace{16pt}

During startup, the KERNAL activates RAM bank 1 as the default for the user.\\

\section{I/O Area}

This is the memory map of the I/O Area:\\

\begin{tabular}{|c|p{0.60\linewidth}|c|}
	\hline
	{\bfseries Addresses} & {\bfseries Description} & {\bfseries Speed}\\ \hline
	\$9F00-\$9F0F & VIA I/O controller \#1                 & 8 MHz\\ \hline
	\$9F10-\$9F1F & VIA I/O controller \#2                 & 8 MHz\\ \hline
	\$9F20-\$9F3F & VERA video controller                 & 8 MHz\\ \hline
	\$9F40-\$9F41 & YM2151 audio controller               & 2 MHz\\ \hline
	\$9F42-\$9F5F & Unavailable                           &  --- \\ \hline
	\$9F60-\$9F7F & Expansion Card Memory Mapped IO3      & 8 MHz\\ \hline
	\$9F80-\$9F9F & Expansion Card Memory Mapped IO4      & 8 MHz\\ \hline
	\$9FA0-\$9FBF & Expansion Card Memory Mapped IO5      & 2 MHz\\ \hline
	\$9FC0-\$9FDF & Expansion Card Memory Mapped IO6      & 2 MHz\\ \hline
	\$9FE0-\$9FFF & Cartidge/Expansion Memory Mapped IO7  & 2 MHz\\ \hline
\end{tabular}

\subsection{Expansion Cards and Cartridges}

Expansion cards can be accessed via memory-mapped I/O (MMIO), as well as I2C.
Cartridges are essentially expansion cards which are housed in an external
enclosure and may contain RAM, ROM and an I2C EEPOM (for save data). Internal
expansion cards may also use the RAM/ROM space, though this could cause
conflicts.\\

While they may be uncomon, since cartridges are essentially external expansion
cards in a shell, that means they can also use MMIO. This is only necessary
when a cartridge includes some sort of hardware expansion and MMIO was desired
(as opposed to using the I2C bus). In that case, it is recommended cartridges
use the IO7 range and that range should be the last option used by expansion
cards in the system.\\

\emph{MMIO is unneeded for cartridges which simply have RAM/ROM.}\\

