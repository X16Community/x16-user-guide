%----------------------------------------------------------------------------------------
%	PART - Getting to Know Your Commander X16
%----------------------------------------------------------------------------------------

\makeatletter\@openrightfalse
\part{Getting to Know Your Commander X16}

\outputtypein{
	\keybackgroundcolor{gray}
	\keytextcolor{black}
	1 PRINT "X16" \widekey{return}\\\\
	2 GOTO 1 \widekey{return}\\\\
	\keybackgroundcolor{black}
	\keytextcolor{white}
	\key{R}\key{U}\key{N}
	\keybackgroundcolor{gray}
	\keytextcolor{black}
	\widekey{return}
	}

\begin{tikzpicture}
	\hyphenpenalty=10000
	\bubble{2.5in}{2.5in}{5.2}{5.8}{
		This line tells the X16 to print what's between the quotation marks.}
	\bubble{2.5in}{1.75in}{4.3}{4.5}{
		This line tells the X16 to go back to Line 1 and print it again.}
	\bubble{2.5in}{1.1in}{5.0}{3.0}{
		Typing the word RUN makes the program run.}
\end{tikzpicture}

%----------------------------------------------------------------------------------------
%	CHAPTER - Getting Started
%----------------------------------------------------------------------------------------

\chapter{Getting Started}\index{Sectioning}


Congratulations!  Your Command X16 is up and running and ready to accept your
first commands.  When it starts, it should display a message at the top letting
you know that it is running BASIC and how much memory is available.  There will
also be a white blinking rectangle called a \emph{cursor}.  This is how the X16
signals that it is waiting for you.

\section{The Start Screen}

\screenbox{2.75in}{2in}{
	**** X16 BASIC ****\\
	512k HIGH RAM\\
	38655 BASIC BYTES FREE\\\\
	READY.\\
	\cursor
}

\begin{tikzpicture}
	\hyphenpenalty=10000
	\bubble{2.5in}{1.0in}{1.1}{2.5}{
		This is the X16 saying it's ready.}
\end{tikzpicture}


\begin{tcolorbox}[
		colback=white,
		colframe=blue,
		width=4.25in,
		height=2.1in,
		]

		\keybackgroundcolor{gray}
		\keytextcolor{black}

		{\sffamily\bfseries\fontsize{16pt}{16pt}\selectfont X16 TIP:\\}

		If you type a character on the screen that you don't want, press the
		\widekey{backspace} key.  This key will erase the character immediately
		to the left of the cursor.\\

		Use this key as often as you like to delete unwanted characters.

\end{tcolorbox}

\section{Experiment a little}

\keybackgroundcolor{white}
\keytextcolor{black}

It's time to start pressing keys and giving your Commander X16 something to do!
Press the following keys:\\

\key{p} \key{r} \key{i} \key{n} \key{t}\\



As you press each key, the cursor moves to the right.  The cursor will always
show you where the next character will be typed.  Next, locate one of the
\keybackgroundcolor{gray}\widekey{shift} keys on the keyboard.  There will be
one on the right and one on the left, but they both do the same thing: modify another
key when pressed at the same time as \widekey{shift}.

%----------------------------------------------------------------------------------------
%	CHAPTER - Your First Computer Program
%----------------------------------------------------------------------------------------

\chapter{Your First Computer Program}\index{Sectioning}

\@openrighttrue\makeatother
